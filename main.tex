\documentclass[a4paper,10pt]{article}
\usepackage{latexsym}
\usepackage{enumitem}
\usepackage{titlesec}
\usepackage{geometry}
\usepackage{fancyhdr}
\usepackage{tikz}
\usepackage{hyperref}
\geometry{top=2mm, bottom=2mm, left=2mm, right=2mm}
\setlist{nolistsep}
\renewcommand{\labelitemi}{$\vcenter{\hbox{\tiny$\bullet$}}$}
\pagenumbering{gobble}

% Custom section spacing
\titlespacing\section{0pt}{2pt plus 2pt minus 2pt}{2pt plus 1pt minus 1pt}
\titlespacing\subsection{0pt}{2pt plus 1pt minus 1pt}{2pt plus 1pt minus 1pt}

% Custom section formatting
\titleformat{\section}{\large\bfseries}{}{0em}{}
\titleformat{name=\section,numberless}{\large\bfseries}{}{0em}{}[\titlerule]


\begin{document}

\begin{center}
    {\LARGE \bfseries Srinivas Peri}\\
    +1 (857)-675-0644 | \texttt{peri.sr@northeastern.edu}|
    \href{https://www.linkedin.com/in/srinivas-peri-yob1998}{LinkedIn} |\href{https://github.com/MAHASHANA/}{GitHub}
\end{center}

\vspace{0.5ex}

\section*{Professional Summary}
Highly motivated Master of Science in Robotics graduate with extensive hands-on experience in robotics, computer vision, and autonomous systems. Demonstrated expertise in developing AI-driven systems, enhancing simulation environments, and implementing advanced machine learning algorithms for object detection and navigation. Proven ability to lead projects from conception to deployment, seeking a challenging role to leverage my skills in advanced robotics and AI solutions.

\section*{Education}
\noindent\textbf{Northeastern University, Boston MA} \hfill Dec 2024 \\
Master of Science in Robotics (ECE) | GPA: 3.7/4 \\
Coursework: Robotic Sensing and Navigation, Reinforcement Learning, Computer Vision and Pattern Recognition, Mobile Robotics, Machine Learning, and Reinforcement Learning

\noindent\textbf{Sreenidhi Institute of Science and Technology, Hyderabad India} \hfill Jun 2020 \\
Bachelor of Technology in Electronics and Communications Engineering | GPA: 8.15/10 \\
Coursework: Digital Signal Processing, Object Oriented Programming, Microprocessors and Microcontrollers

\section*{Experience}
\noindent\textbf{Graduate Research Assistant, Northeastern University | Boston, MA} \hfill Apr 2024 – Present
\begin{itemize}
    \item Spearheading the development of drones designed to navigate GPS-denied environments through advanced Visual-Inertial Odometry (VIO) techniques, significantly improving autonomous flight capabilities.
    \item Successfully implemented \textbf{RTAB SLAM} in \textbf{ROS2}, integrating \textbf{sensor fusion} with the D435 depth camera and flight controller IMU, enhancing the accuracy and reliability of spatial mapping in complex environments.
    \item Currently focused on advancing autonomous capabilities, including automated takeoff, precise waypoint navigation, and the development of automatic takeoff and landing systems for dynamic ground vehicles.
\end{itemize}

\noindent\textbf{Research Intern (MBSE \& Autonomous Driving), TWT.GmbH | Munich, Germany} \hfill Jun 2023 – Dec 2023
\begin{itemize}
    \item Led the innovation of a CNN-based AI system for precise rain simulation, seamlessly integrating it with \textbf{Blender’s particle system}. Developed an \textbf{LSTM model} to predict \textbf{Lidar} noise under varying rain intensities, significantly enhancing the realism and precision of weather impact simulations, thereby improving Lidar data accuracy in diverse scenarios.
    \item Played a key role in the \textbf{SUMO} and \textbf{Unreal Engine 5} integration project, emphasizing \textbf{georeferencing} for accurate real-world geography mapping. Enhanced the Tronis-Sumo plugin by addressing critical errors and optimizing code, resulting in substantial advancements in \textbf{traffic simulation realism}.
\end{itemize}

\noindent\textbf{Assistant System Engineer, TATA Consultancy Services | Hyderabad, India} \hfill Apr 2021 – Jun 2022
\begin{itemize}
    \item Engineered \textbf{sophisticated Splunk queries} and dashboards to enable real-time \textbf{data analysis} and tracking, enhancing operational efficiency and decision-making processes.
    \item Collaborated with a 16-member team to efficiently address client requests for log retrieval and analysis, providing critical support across the Americas and Europe regions.
\end{itemize}

\noindent\textbf{IOT Developer, Robic Rufarm | Hyderabad, India} \hfill Oct 2020 – Feb 2021
\begin{itemize}
    \item Developed innovative software solutions for monitoring pH and oxygen levels in aquaculture, incorporating real-time data visualization tools optimized for cloud deployment.
\end{itemize}

\noindent\textbf{Drones and Robotics Research Intern, Activa Software.Inc  | Hyderabad, India} \hfill Jun 2019 – Jul 2020
\begin{itemize}
    \item Engineered a GPS-equipped autonomous UAV by integrating the u-blox GPS with the Pixhawk PX4 flight controller, achieving significant advancements in autonomous navigation.
    \item Attained autonomous navigation by programming the UAV to traverse predetermined checkpoints utilizing the open-source Ardupilot autopilot system.
    \item Implemented a 3D LIDAR-LOAM approach to generate high-quality point clouds of large-scale structures, greatly enhancing virtual reality visualization accuracy and detail.
\end{itemize}

\section*{Academic Research \& Projects}
\noindent\textbf{Autonomous Vehicle Navigation with DQN in Carla | TensorFlow, OpenCV, UnrealEngine} \hfill Apr 2024 – May 2024
\begin{itemize}
    \item Designed and developed a DQN-based autonomous vehicle simulation in CARLA, enabling the vehicle to consistently navigate at 50 mph while avoiding obstacles, demonstrating robust autonomous driving capabilities.
\end{itemize}

\noindent\textbf{Object Detection Using Lidar Point Clouds | Open3d ML} \hfill Dec 2022 – Dec 2022
\begin{itemize}
    \item Conducted extensive research and implementation of 3D-object detection techniques using VLP-16 LiDAR point cloud data, enhancing object detection accuracy.
    \item Improved object detection efficiency by implementing voxel down-sampling (reducing 26,000 points to 8,000) and utilizing the DBSCAN clustering technique from Open-3DML to accurately define object shapes and draw bounding boxes.
\end{itemize}

\section*{Skills}
\begin{itemize}
    \item \textbf{Programming Languages}: Python, C++, MATLAB
    \item \textbf{Operating Systems}: Linux, Windows
    \item \textbf{Hardware Expertise}: Raspberry Pi, Nvidia Jetson Boards, Analog and Digital Sensors, Arduino
    \item \textbf{Technologies}: Robotics, Drones, Computer Vision, Neural Networks, Visual \& Lidar SLAM, IoT
    \item \textbf{Software Tools}: ROS/ROS2, Gazebo, MeshLab, GitHub, Unreal Engine, Splunk, Blender
    \item \textbf{Communication Protocols}: UART, I2C, MSP, MAVLink 
    \item \textbf{Libraries and Frameworks}: Pandas, NumPy, SciPy, Matplotlib, Scikit-Learn, OpenCV, CUDA, TensorFlow, Blender Python (Bpy)
\end{itemize}

\end{document}
