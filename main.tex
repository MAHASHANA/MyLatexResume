\documentclass[a4paper,10pt]{article}
\usepackage{latexsym}
\usepackage{enumitem}
\usepackage{titlesec}
\usepackage{geometry}
\usepackage{fancyhdr}
\usepackage{tikz}
\usepackage{hyperref}
\geometry{top=2mm, bottom=2mm, left= 5mm, right= 5mm}
\setlist{nolistsep}
\renewcommand{\labelitemi}{$\vcenter{\hbox{\tiny$\bullet$}}$}
\pagenumbering{gobble}

% Custom section spacing
\titlespacing\section{0pt}{1pt plus 1pt minus 1pt}{1pt plus 1pt minus 1pt}
\titlespacing\subsection{0pt}{1pt plus 1pt minus 1pt}{1pt plus 1pt minus 1pt}

% Custom section formatting
\titleformat{\section}{\large\bfseries}{}{0em}{}
\titleformat{name=\section,numberless}{\large\bfseries}{}{0em}{}[\titlerule]


\begin{document}

\begin{center}
    {\LARGE \bfseries Srinivas Peri}\\
    +1 (857)-675-0644 | {\href{https://outlook.office.com/mail/}{peri.sr@northeastern.edu}}|
    \href{https://www.linkedin.com/in/srinivas-peri-yob1998}{LinkedIn} |\href{https://github.com/MAHASHANA/}{GitHub}
\end{center}

\vspace{0.5ex}

\section*{Education}
\noindent\textbf{Northeastern University, Boston MA} \hfill Dec 2024 \\
Master of Science in Robotics (ECE) | GPA: 3.7/4 \\
Coursework: Robotic Sensing and Navigation, Reinforcement Learning, Computer Vision and Pattern Recognition, Mobile Robotics, Machine Learning, and Reinforcement Learning

\noindent\textbf{Sreenidhi Institute of Science and Technology, Hyderabad India} \hfill Jun 2020 \\
Bachelor of Technology in Electronics and Communications Engineering | GPA: 8.15/10 \\
Coursework: Digital Signal Processing, Object Oriented Programming, Microprocessors and Microcontrollers

\section*{Experience}
\noindent\textbf{Graduate Research Assistant, Northeastern University | Boston, MA} \hfill Apr 2024 – Present
\begin{itemize}
    \item Spearheading the development of drones designed to navigate GPS-denied environments through advanced Visual-Inertial Odometry (VIO) techniques, significantly improving autonomous flight capabilities.
    \item Successfully implemented \textbf{RTAB SLAM} in \textbf{ROS2}, integrating \textbf{sensor fusion} with the D435 depth camera and flight controller IMU, enhancing the accuracy and reliability of spatial mapping in complex environments.
    \item Currently focused on advancing autonomous capabilities, including automated takeoff, precise waypoint navigation, and the development of automatic takeoff and landing systems for dynamic ground vehicles.
\end{itemize}

\noindent\textbf{Research Intern MBSE and Autonomous Driving, TWT Gmbh | Munich, Germany} \hfill Jun 2023 – Dec 2023

    \noindent\textbf{Project: Realistic Rain Simulation}
\begin{itemize}
    \item Led the development of a CNN-based AI system for rain simulation, integrating it with Blender’s particle system.
    \item Developed an \textbf{LSTM} model to predict Lidar noise under varying rain intensities, improving weather impact simulation realism. Achieved a \textbf{classification} \textbf{model} with \textbf{92\%} accuracy by uncovering relationships between rain intensity, Lidar noise, and image-based intensity maps.
    \item Integrated multiple sensor data (Lidar, disdrometer, and RGB) to enhance rain intensity predictions and predict the level of loss in point cloud data, based on the methodologies from this  \href{https://ieeexplore.ieee.org/stamp/stamp.jsp?tp=&arnumber=10011628}{publication}.
\end{itemize}
\noindent\textbf {Project: SaveNow}
\begin{itemize}
    \item Key contributor in SUMO and Unreal Engine 5 integration, focusing on georeferencing for accurate real-world mapping.
    \item Enhanced the Tronis-Sumo plugin, improving traffic simulation realism for the city of Ingolstadt.
\end{itemize}

\noindent\textbf{Assistant System Engineer, TATA Consultancy Services | Hyderabad, India} \hfill Apr 2021 – Jun 2022
\begin{itemize}
    \item Engineered \textbf{sophisticated Splunk queries} and dashboards to enable real-time \textbf{data analysis} and tracking, enhancing operational efficiency and decision-making processes by 30\%.
    \item Collaborated with a 16-member team to efficiently address client requests for log retrieval and analysis, providing critical support across the Americas and Europe regions.
\end{itemize}
\noindent\textbf{IOT Developer, Robic Rufarm | Hyderabad, India} \hfill Oct 2020 – Feb 2021
\begin{itemize}
    \item Developed innovative software solutions for monitoring pH and oxygen levels in aquaculture, incorporating real-time data visualization tools optimized for cloud deployment.
\end{itemize}

\noindent\textbf{Drones and Robotics Research Intern, Activa Software.Inc  | Hyderabad, India} \hfill Jun 2019 – Jul 2020
\begin{itemize}
    \item Engineered a GPS-equipped autonomous UAV by integrating the u-blox GPS with px4 autopilot and Pixhawk flight controller, achieving significant advancements in autonomous navigation.
    \item Attained autonomous navigation by programming the UAV to traverse predetermined checkpoints utilizing the open-source Ardupilot autopilot system.
    \item Implemented 3D LIDAR-LOAM approach to generate high-quality point clouds of large-scale structures using ROS, and greatly enhancing virtual reality visualization accuracy and detail.
\end{itemize}

\section*{Projects}
\noindent\textbf{Autonomous Vehicle Navigation with DQN in Carla | TensorFlow, OpenCV, UnrealEngine} \hfill Apr 2024 – May 2024
\begin{itemize}
    \item Designed and developed a DQN-based autonomous vehicle simulation in CARLA, achieving robust navigation at 50 mph while avoiding obstacles.
    \item Demonstrated strong autonomous driving capabilities with the vehicle consistently navigating complex environments.
    \item Currently refining the system to reach the optimal policy by adding rules and modifications, addressing the current suboptimal policy where the vehicle sometimes follows circular paths.
\end{itemize}

\noindent\textbf{Object Detection Using Lidar Point Clouds | Open3d ML} \hfill Dec 2022 – Dec 2022
\begin{itemize}
    \item Conducted extensive research and implementation of 3D-object detection techniques using VLP-16 LiDAR point cloud data, enhancing object detection accuracy.
    \item Improved object detection efficiency by implementing voxel down-sampling (reducing 26,000 points to 8,000) and utilizing the DBSCAN clustering technique from Open-3DML to accurately define object shapes and draw bounding boxes.
\end{itemize}
\section*{Publications}
\begin{itemize}
    \item "LiDAR-Driven Rain Intensity Predictions: A Robust Classifier Modelling Approach." Srinivas Peri et al. Pending submission to an international conference.
\end{itemize}

\section*{Skills}
{\fontsize{10pt}{10pt}\selectfont  % Start of custom font size
\begin{itemize}
    \item \textbf{Programming Languages}: Python, C++, MATLAB
    \item \textbf{Operating Systems}: Linux, Windows
    \item \textbf{Hardware Expertise}: Raspberry Pi, Nvidia Jetson Boards, Analog and Digital Sensors, Arduino
    \item \textbf{Technologies}: Robotics, Drones, Computer Vision, Neural Networks, Visual \& Lidar SLAM, Sensor Fusion, IoT
    \item \textbf{Software Tools}: ROS/ROS2, Gazebo, MeshLab, GitHub, Unreal Engine, Splunk, Blender
    \item \textbf{Communication Protocols}: UART, I2C, MSP, MAVLink 
    \item \textbf{Libraries and Frameworks}: Pandas, NumPy, SciPy, Matplotlib, Scikit-Learn, OpenCV, CUDA, TensorFlow
\end{itemize}
}  % End of custom font size

\end{document}
